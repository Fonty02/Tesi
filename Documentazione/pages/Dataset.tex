\section{Dataset del regressore}
In questo capitolo si analizza il dataset utilizzato e come questo è stato trattato per l'addestramento dei modelli. Inoltre, si descrivono le feature di input e output del modello.

\noindent Il dataset nella sua totalità è composto da 13 feature di input e una feature di output. Le feature di input possiamo suddividerle in 4 categorie:
\begin{itemize}
    \item \textbf{Feature relative al dataset}, quali \textit{n\_users}, \textit{n\_items}, \textit{n\_inter}, \textit{sparsity}
    \item \textbf{Feature relative al knowledge graph}, quali \textit{kg\_entities}, \textit{kg\_relations}, \textit{kg\_triples}, \textit{kg\_items}
    \item \textbf{Feature relative all'hardware utilizzato per l'addestramento}, quali \textit{cpu\_cores}, \textit{ram\_size}, \textit{is\_gpu}
    \item \textbf{Feature relative al modello}, quali \textit{model\_name}, \textit{model\_type}
\end{itemize}
Nel dataset sono presenti 201 righe (dunque 201 esperimenti distinti).
\subsection{Descrizione delle feature di output}
La feature di output \textit{emissions} rappresenta le emissioni di CO$_2$eq prodotte dalla macchina durante l'addestramento del modello.
\subsection{Descrizione delle feature di input}
\begin{center}
\begin{table}[H]
    \centering
    \begin{tabularx}{\textwidth}{|c|X|}
        \hline
        \textbf{Feature} & \textbf{Descrizione} \\
        \hline
        n\_users & Numero di utenti presenti nel dataset \\
        \hline
        n\_items & Numero di items presenti nel dataset \\
        \hline
        n\_inter & Numero di interazioni nel dataset. Per interazione si intendono le varie interazioni (valutazioni) tra gli utenti nel dataset e gli item nel dataset \\
        \hline
        sparsity & Sparsità del dataset. La sparsità indica la percentuale di valori mancanti nel dataset (quindi mancanza di interazione tra utenti e item)\\
        \hline
        kg\_entities & Numero di entità nel knowledge graph. Un'entità è un oggetto distintivo o un concetto unico all'interno del Knowledge Graph \\
        \hline
        kg\_relations & Numero di relazioni nel knowledge graph. Le relazioni rappresentano i legami o collegamenti tra le entità all'interno del Knowledge Graph. Sono spesso definite dai predicati nelle triple \\
        \hline
        kg\_triples & Numero di triple nel knowledge graph. Una triple è una struttura dati fondamentale nel Knowledge Graph che consiste in tre parti: soggetto, predicato e oggetto. Queste triple rappresentano le relazioni tra le entità \\
        \hline
        kg\_items & Numero di items nel knowledge graph. Gli "Items" nel contesto del Knowledge Graph sono gli oggetti specifici o le entità che sono inclusi nel grafo \\
        \hline
        cpu\_cores & Numero di core della CPU \\
        \hline
        ram\_size & Dimensione della RAM \\
        \hline
        is\_gpu & Booleano che indica se la macchina ha usato una GPU per l'addestramento \\
        \hline
        model\_name & Nome del modello \\
        \hline
        model\_type & Tipo del modello \\
        \hline
    \end{tabularx}
    \caption*{Descrizione delle feature di input}
\end{table}
\end{center}

Per quanto riguarda la feature \textit{model\_type} abbiamo i seguenti valori:
\begin{itemize}
    \item \textbf{General}: Modelli che si basano su tecniche tradizionali
    \item \textbf{Knowledge}: Modelli che incorporano conoscenza esterna (knowledge graph) per migliorare le raccomandazioni
\end{itemize}

Per quanto riguarda la feature \textit{model\_name} abbiamo i seguenti valori:
\begin{itemize}
    \item \textbf{BPR} \cite{BPR}: General
    \item \textbf{CDAE} \cite{CDAE}: General
    \item \textbf{CFKG} \cite{CFKG}: Knowledge
    \item \textbf{CKE} \cite{CKE}: Knowledge
    \item \textbf{DGCF} \cite{DGCF}: Knowledge
    \item \textbf{DMF} \cite{DMF}: General
    \item \textbf{DiffRec} \cite{DiffRec}: General
    \item \textbf{ENMF} \cite{ENMF}: General 
    \item \textbf{FISM} \cite{FISM}: General
    \item \textbf{GCMC} \cite{GCMC}: General
    \item \textbf{ItemKNN} \cite{ItemKNN}: General
    \item \textbf{KGCN} \cite{KGCN}: Knowledge
    \item \textbf{KGIN}: \cite{KGIN} Knowledge
    \item \textbf{KGNNLS} \cite{KGNNLS}: Knowledge
    \item \textbf{KTUP} \cite{KTUP}: Knowledge
    \item \textbf{LDiffRec} \cite{LDiffRec}: General
    \item \textbf{LINE} \cite{LINE}: General
    \item \textbf{LightGCN} \cite{LightGCN}: General
    \item \textbf{MKR} \cite{MKR}: Knowledge
    \item \textbf{MacridVAE} \cite{MacridVAE}: General
    \item \textbf{MultiDAE} \cite{MultiDAE}: General
    \item \textbf{MultiVAE} \cite{MultiVAE}: General
    \item \textbf{NCEPLRec} \cite{NCEPLRec}: General
    \item \textbf{NCL} \cite{NCL}: General
    \item \textbf{NGCF} \cite{NGCF}: General
    \item \textbf{NeuMF} \cite{NeuMF}: General
    \item \textbf{Pop}: General
    \item \textbf{Random}: General
    \item \textbf{RecVAE} \cite{RecVAE}: General
    \item \textbf{RippleNet} \cite{RippleNet}: Knowledge
    \item \textbf{SGL} \cite{SGL}: General
    \item \textbf{SLIMElastic} \cite{SLIMElastic}: General
    \item \textbf{SimpleX} \cite{SimpleX}: General
    \item \textbf{SpectralCF} \cite{SpectralCF}: General
    \item \textbf{EASE} \cite{EASE}: General
    \item \textbf{NAIS} \cite{NAIS}: General
    \item \textbf{ADMMSLIM} \cite{ADMMSLIM}: General
    \item \textbf{ConvNCF} \cite{ConvNCF}: General
    \item \textbf{NNCF} \cite{NNCF}: General
\end{itemize}

Altri valori unici presenti per le varie feature di input sono:
\begin{itemize}
    \item \textbf{n\_users}: [22155, 23679, 6040]
    \item \textbf{n\_items}: [54458, 4414, 3706]
    \item \textbf{n\_inter}: [1465871, 1048575, 1000209]
    \item \textbf{sparsity}: [0.99878504, 0.98996762, 0.95531637]
    \item \textbf{kg\_entities}: [26315, 0, 79347]
    \item \textbf{kg\_relations}: [16, 0, 49]
    \item \textbf{kg\_triples}: [96476, 0, 385923]
    \item \textbf{kg\_items}: [11446, 0, 3655]
    \item \textbf{cpu\_cores}: [12, 4]
    \item \textbf{ram\_size}: [64, 16, 27.40581512]
    \item \textbf{is\_gpu}: [1, 0]
\end{itemize}
\subsection{Pre-Processing}

Per poter sfruttare le feature di input per l'addestramento del modello, è stato necessario effettuare un pre-processing. Le feature \textit{model\_name} e \textit{model\_type} sono state trasformate in variabili numeriche. In particolare il valore \textit{general} è stato trasformato in 0 e il valore \textit{knowledge} è stato trasformato in 1.
Per quanto riguarda la feature \textit{model\_name}, ogni valore è stato trasformato in un numero intero univoco. In questo modo, il modello può sfruttare queste feature per l'addestramento.
Prima di cominciare con l'addestramento dei modelli la feature di output è stata separata dalle feature di input. I dati sono poi stati suddivisi rispettivamente in training set e test set. In particolare il 70\% dei dati è stato usato per l'addestramento, mentre il 30\% è stato usato per la valutazione. Inoltre, mediante il \textit{random\_state}=2, è garantita la riproducibilità dell'esperimento.

